% Options for packages loaded elsewhere
\PassOptionsToPackage{unicode}{hyperref}
\PassOptionsToPackage{hyphens}{url}
%
\documentclass[
  twocolumn]{article}
\usepackage{amsmath,amssymb}
\usepackage{lmodern}
\usepackage{iftex}
\ifPDFTeX
  \usepackage[T1]{fontenc}
  \usepackage[utf8]{inputenc}
  \usepackage{textcomp} % provide euro and other symbols
\else % if luatex or xetex
  \usepackage{unicode-math}
  \defaultfontfeatures{Scale=MatchLowercase}
  \defaultfontfeatures[\rmfamily]{Ligatures=TeX,Scale=1}
\fi
% Use upquote if available, for straight quotes in verbatim environments
\IfFileExists{upquote.sty}{\usepackage{upquote}}{}
\IfFileExists{microtype.sty}{% use microtype if available
  \usepackage[]{microtype}
  \UseMicrotypeSet[protrusion]{basicmath} % disable protrusion for tt fonts
}{}
\makeatletter
\@ifundefined{KOMAClassName}{% if non-KOMA class
  \IfFileExists{parskip.sty}{%
    \usepackage{parskip}
  }{% else
    \setlength{\parindent}{0pt}
    \setlength{\parskip}{6pt plus 2pt minus 1pt}}
}{% if KOMA class
  \KOMAoptions{parskip=half}}
\makeatother
\usepackage{xcolor}
\usepackage[margin=1in]{geometry}
\usepackage{color}
\usepackage{fancyvrb}
\newcommand{\VerbBar}{|}
\newcommand{\VERB}{\Verb[commandchars=\\\{\}]}
\DefineVerbatimEnvironment{Highlighting}{Verbatim}{commandchars=\\\{\}}
% Add ',fontsize=\small' for more characters per line
\usepackage{framed}
\definecolor{shadecolor}{RGB}{248,248,248}
\newenvironment{Shaded}{\begin{snugshade}}{\end{snugshade}}
\newcommand{\AlertTok}[1]{\textcolor[rgb]{0.94,0.16,0.16}{#1}}
\newcommand{\AnnotationTok}[1]{\textcolor[rgb]{0.56,0.35,0.01}{\textbf{\textit{#1}}}}
\newcommand{\AttributeTok}[1]{\textcolor[rgb]{0.77,0.63,0.00}{#1}}
\newcommand{\BaseNTok}[1]{\textcolor[rgb]{0.00,0.00,0.81}{#1}}
\newcommand{\BuiltInTok}[1]{#1}
\newcommand{\CharTok}[1]{\textcolor[rgb]{0.31,0.60,0.02}{#1}}
\newcommand{\CommentTok}[1]{\textcolor[rgb]{0.56,0.35,0.01}{\textit{#1}}}
\newcommand{\CommentVarTok}[1]{\textcolor[rgb]{0.56,0.35,0.01}{\textbf{\textit{#1}}}}
\newcommand{\ConstantTok}[1]{\textcolor[rgb]{0.00,0.00,0.00}{#1}}
\newcommand{\ControlFlowTok}[1]{\textcolor[rgb]{0.13,0.29,0.53}{\textbf{#1}}}
\newcommand{\DataTypeTok}[1]{\textcolor[rgb]{0.13,0.29,0.53}{#1}}
\newcommand{\DecValTok}[1]{\textcolor[rgb]{0.00,0.00,0.81}{#1}}
\newcommand{\DocumentationTok}[1]{\textcolor[rgb]{0.56,0.35,0.01}{\textbf{\textit{#1}}}}
\newcommand{\ErrorTok}[1]{\textcolor[rgb]{0.64,0.00,0.00}{\textbf{#1}}}
\newcommand{\ExtensionTok}[1]{#1}
\newcommand{\FloatTok}[1]{\textcolor[rgb]{0.00,0.00,0.81}{#1}}
\newcommand{\FunctionTok}[1]{\textcolor[rgb]{0.00,0.00,0.00}{#1}}
\newcommand{\ImportTok}[1]{#1}
\newcommand{\InformationTok}[1]{\textcolor[rgb]{0.56,0.35,0.01}{\textbf{\textit{#1}}}}
\newcommand{\KeywordTok}[1]{\textcolor[rgb]{0.13,0.29,0.53}{\textbf{#1}}}
\newcommand{\NormalTok}[1]{#1}
\newcommand{\OperatorTok}[1]{\textcolor[rgb]{0.81,0.36,0.00}{\textbf{#1}}}
\newcommand{\OtherTok}[1]{\textcolor[rgb]{0.56,0.35,0.01}{#1}}
\newcommand{\PreprocessorTok}[1]{\textcolor[rgb]{0.56,0.35,0.01}{\textit{#1}}}
\newcommand{\RegionMarkerTok}[1]{#1}
\newcommand{\SpecialCharTok}[1]{\textcolor[rgb]{0.00,0.00,0.00}{#1}}
\newcommand{\SpecialStringTok}[1]{\textcolor[rgb]{0.31,0.60,0.02}{#1}}
\newcommand{\StringTok}[1]{\textcolor[rgb]{0.31,0.60,0.02}{#1}}
\newcommand{\VariableTok}[1]{\textcolor[rgb]{0.00,0.00,0.00}{#1}}
\newcommand{\VerbatimStringTok}[1]{\textcolor[rgb]{0.31,0.60,0.02}{#1}}
\newcommand{\WarningTok}[1]{\textcolor[rgb]{0.56,0.35,0.01}{\textbf{\textit{#1}}}}
\usepackage{graphicx}
\makeatletter
\def\maxwidth{\ifdim\Gin@nat@width>\linewidth\linewidth\else\Gin@nat@width\fi}
\def\maxheight{\ifdim\Gin@nat@height>\textheight\textheight\else\Gin@nat@height\fi}
\makeatother
% Scale images if necessary, so that they will not overflow the page
% margins by default, and it is still possible to overwrite the defaults
% using explicit options in \includegraphics[width, height, ...]{}
\setkeys{Gin}{width=\maxwidth,height=\maxheight,keepaspectratio}
% Set default figure placement to htbp
\makeatletter
\def\fps@figure{htbp}
\makeatother
\setlength{\emergencystretch}{3em} % prevent overfull lines
\providecommand{\tightlist}{%
  \setlength{\itemsep}{0pt}\setlength{\parskip}{0pt}}
\setcounter{secnumdepth}{-\maxdimen} % remove section numbering
\ifLuaTeX
  \usepackage{selnolig}  % disable illegal ligatures
\fi
\IfFileExists{bookmark.sty}{\usepackage{bookmark}}{\usepackage{hyperref}}
\IfFileExists{xurl.sty}{\usepackage{xurl}}{} % add URL line breaks if available
\urlstyle{same} % disable monospaced font for URLs
\hypersetup{
  pdftitle={Exam QFI},
  pdfauthor={Steven},
  hidelinks,
  pdfcreator={LaTeX via pandoc}}

\title{Exam QFI}
\author{Steven}
\date{2023-04-28}

\begin{document}
\maketitle

{
\setcounter{tocdepth}{2}
\tableofcontents
}
\newpage

\hypertarget{math-of-financial-derivatives}{%
\section{Math of Financial
Derivatives}\label{math-of-financial-derivatives}}

~~~~~~\textbf{Financial Derivatives} are securities that derive their
value from cash markets. In Ingersoll's definition, a financial contract
is a derivative security.\\
\hspace*{0.333em}\hspace*{0.333em}\hspace*{0.333em}\hspace*{0.333em}\hspace*{0.333em}\hspace*{0.333em}\textbf{Cash
and Carry Markets}: Gold, silver, T-bonds. The idea is to borrow cash,
buy asset, and hold until expiration.\\
\hspace*{0.333em}\hspace*{0.333em}\hspace*{0.333em}\hspace*{0.333em}\hspace*{0.333em}\hspace*{0.333em}\textbf{Forward
Contract}: obligation to buy/sell an asset at a specified price at a
future date. if cash price is higher at expiration then forward, and
long position make a profit.

\begin{tabular}{l|l|l}
\hline
Characteristic & Futures & Forwards\\
\hline
Mark to Market & Yes & No\\
\hline
Settlement & Daily & Upon Maturity\\
\hline
Margin System & Yes & No\\
\hline
Trading & Exchanges & OTC\\
\hline
Types & Standardized & Custom\\
\hline
Flexibility & Less & More\\
\hline
Liquidity & More & Less\\
\hline
Counterparty Risk & Less & More\\
\hline
\end{tabular}

~~~~~~\textbf{Repurchase Agreement (REPO)}: A transaction in which one
party sells a security in return for cash, with an agreement to
repurchase the security at an agreed upon price at a future date.\\
\hspace*{0.333em}\hspace*{0.333em}\hspace*{0.333em}\hspace*{0.333em}\hspace*{0.333em}\hspace*{0.333em}\textbf{Options}:
\n

\begin{itemize}
\tightlist
\item
  European: The right to buy \(S_t\) at strike k. This right may be
  excercised at time T. Where \(S_T <K\) is out the money and
  \(S_T > K\) is in the money. \[C_T = \max(S_T -K)\]
\end{itemize}

~~~~~~\textbf{Swaps}: Exchange of one set of cashflows for another.
Swaps are simultaneous and can involve curruncies, interst rates, etc.
Swaps can be priced through forwards. Cancelable swaps allow the right
to cancel swap contracts.\\
\hspace*{0.333em}\hspace*{0.333em}\hspace*{0.333em}\hspace*{0.333em}\hspace*{0.333em}\hspace*{0.333em}\textbf{Arbitrage}:
Taking simultaneous positions so that it guarantees a riskless profit
higher than the risk-free rate. Let \(d_{ij}\) be the number of units of
account paid by one unit of security i in state j. Rows represent asset,
columns represent state (interest, conflict, recession, etc\ldots)

\[
\text{Asset Values} = S_T = 
\begin{bmatrix} 
S_1(t) \\
S_2(t) \\
\vdots \\
S_n(t)
\end{bmatrix}
\]

\[
D_t = \begin{bmatrix}
d_{11} & d_{12} & \cdots & d_{1k} \\
d_{21} & d_{22} & & d_{2k} \\
\vdots & & \ddots & \vdots\\
d_{i1} & d_{i2} & \cdots & d_{ik}
\end{bmatrix}
\]

\[
\text{example of a 2 state portfolio with cash, security, and call}\\ \begin{bmatrix}
(1+r\Delta) & (1+r\Delta) \\
S_1(t + \Delta) & S_2(t + \Delta) \\
C_1(t + \Delta) & C_2(t + \Delta) 
\end{bmatrix}
\]

No arbitrage possibility exists iff:

\[
\begin{bmatrix}
1 \\ S(t) \\ C(t)
\end{bmatrix} 
=
\begin{bmatrix}
(1+r\Delta) & (1+r\Delta) \\
S_1(t + \Delta) & S_2(t + \Delta) \\
C_1(t + \Delta) & C_2(t + \Delta) 
\end{bmatrix} 
\begin{bmatrix}
\psi_1 \\ \psi_2
\end{bmatrix} 
\] (\(\psi_n\) represents the risk neutral probability discounted at the
risk free rate)

\begin{Shaded}
\begin{Highlighting}[]
\NormalTok{Hello}
\end{Highlighting}
\end{Shaded}


\end{document}
